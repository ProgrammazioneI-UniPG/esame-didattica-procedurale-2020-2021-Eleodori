\documentclass[addpoints,11pt]{exam}
\usepackage[top=0.5in, bottom=0.5in, left=0.5in, right=0.5in]{geometry}
\usepackage[utf8]{inputenc}
\usepackage{listings}
\usepackage{color,graphicx}
\usepackage{multicol}
\usepackage{MnSymbol}

\definecolor{codegreen}{rgb}{0,0,0}
\definecolor{codegray}{rgb}{0,0,0}
\definecolor{codepurple}{rgb}{0,0,0}
\definecolor{backcolour}{rgb}{1,1,1}

\lstdefinestyle{mystyle}{
	backgroundcolor=\color{backcolour},   
	commentstyle=\color{codegreen},
	keywordstyle=\color{black},
	numberstyle=\tiny\color{codegray},
	stringstyle=\color{codepurple},
	basicstyle=\footnotesize,
	breakatwhitespace=false,         
	breaklines=true,                 
	captionpos=b,                    
	keepspaces=true,                 
	numbers=left,                    
	numbersep=5pt,                  
	showspaces=false,                
	showstringspaces=false,
	showtabs=false,                  
	tabsize=2
}

\lstset{style=mystyle}

\pagestyle{empty}

\begin{document}
 \boxedpoints
 \pointname{~punti}
\begin{center}
\fbox{\fbox{\parbox{7in}{\centering
Esercitazione Programmazione Procedurale - Michele Eleodori - Domenico Pagliuca}}}
\end{center}
 
\vspace{5mm}
 
\noindent\makebox[\textwidth]{Nome e Cognome: \rule{8cm}{.1pt} \hspace{1cm} Matricola:  \rule{5cm}{.1pt}}

\begin{questions} 


% DOMANDA NUMERO 1
\question[2]
Scrivere a cosa serve il seguente codice.

\begin{minipage}[t]{0.5\linewidth}
\begin{lstlisting}[language=C]
int esercizio (int a[ ], int n) {
     int i, x, quanto=0;
     printf("\n Inserisci un numero :");
     scanf("%d", &x);
     for(i=0; i<n; i++) {
       if(x==a[i]) { 
          quanto++;
       }   
     }
     printf("\nquanto = %d", quanto);
return;
}
     
 
\end{lstlisting}
\end{minipage}
\begin{minipage}[t]{0.5\linewidth}
	\makeemptybox{120pt}
\end{minipage}

% DOMANDA NUMERO 2

\question[2]
Scrivere cosa stampa il seguente codice.

\begin{minipage}[t]{0.55\linewidth}
\begin{lstlisting}[language=C]
main() {
   float dividendo = 20;
   float divisore = 2;
   float risultato;
   if( divisore = 3){
      printf("impossibile\n");
      exit(-1);
   }
   risultato = dividendo / divisore;
   printf ("il risultato e': %f\n" , risultato);
   exit(0);
}
\end{lstlisting}
\end{minipage}
\begin{minipage}[t]{0.45\linewidth}
	\makeemptybox{150pt}
\end{minipage}

% DOMANDA NUMERO 3

\question[2]
Cosa stampa il seguente codice? (si consiglia di eseguire il programma ad ogni passo)

\begin{minipage}[t]{0.6\linewidth}
\begin{lstlisting}[language=C]
int  main() {

   int a;
   for (a = 0; a < 10; a++) {
      printf("%d\t\n", ricorsione(a));}
}
int ricorsione(int a) {

   if(a == 0) {
      return 0;}
   if(a == 1) {
      return 1;}
   return ricorsione(a-1) + ricorsione(a-2);}

\end{lstlisting}
\end{minipage}
\begin{minipage}[t]{0.4\linewidth}
	\makeemptybox{120pt}
\end{minipage}
\newpage

\newpage

%DOMANDA NUMERO 4

\question[2]
Scrivere un programmi che calcoli il fattoriale di un numero, il codice da scrivere è corto e necessita di due funzioni (main e una ricorsiva). Si ricorda che il fattoriale di un numero si trova moltiplicando tutti i numeri che lo precedono (escluso lo zero), con esso incluso.\\
\begin{minipage}[t]{1\linewidth}
	\makeemptybox{220pt}
\end{minipage}

%DOMANDA NUMERO 5
\question[2]
Se due puntatori "i" e "j" dello stesso tipo puntano allo stesso indirizzo, allora i == j restituisce vero?.

\begin{oneparcheckboxes}
	\choice Vero\\
	\choice Falso 
\end{oneparcheckboxes}

%DOMANDA NUMERO 6

\question[2]
Considerando il seguente codice:\\

int a = 3, b = 2;\\
int c = a+++b;

L'espressione a+++b

\begin{oneparcheckboxes}
	\choice significa (a++) + b\\
	\choice porta a un errore di compilazione poiché l'operatore +++ non è definito\\
	\choice equivale a + (++b)\\
	\choice invoca un istruzione indefinita
\end{oneparcheckboxes}
 
% DOMANDA NUMERO 7

\question[2]
Il seguente codice è giusto o errato ? \\
\begin{lstlisting}[language=C]
int main()
{
     char str1[20] = "123456";
     printf("Length of string str1: %d", strlen(str1));
     return 0;
}
\end{lstlisting}
\begin{oneparcheckboxes}
	\choice è giusto\\
	\choice restituisce un errore di sintassi in quanto dovrebbe essere "strlenght"\\
	\choice restituisce un errore logico in quanto la variabile si chiama "str" e non "strlen"\\
	\choice restituisce un errore di sintassi in quanto dovrebbe essere "str.lenght"\\
	\choice restituisce un errore di sintassi in quanto dovrebbe essere "str.len"\\
\end{oneparcheckboxes}
\newpage

%DOMANDA NUMERO 8

\question[2]
Cosa stampa il seguente codice ?

\begin{lstlisting}[language=C]
int main(void){
    float
    a = 30.5,
    b = 5.8;

  int risultato = (int) a / (int) b;

  printf("Risultato = %d\n", risultato);
}
\end{lstlisting}
\begin{oneparcheckboxes}
	\choice errore, in quanto la sintassi corretta sarebbe "float a = 30.5"\\
	\choice errore, in quanto non si può spicificare il tipo di varibile all'interno di un operazione\\
	\choice 5,25 ma con un warring poichè si forza un operazione con degli int ereditati da dei float   \\
	\choice stampa "Risultato = 5,25\\
	\choice stampa "Risultato = 5" riportando 25\\
	\choice stampa "Risultato = 6"\\
	\choice stampa "Risultato = 6" ma con un warring\\
\end{oneparcheckboxes}
 
%DOMANDA NUMERO 9

\question[2]
Descrivere a cosa serve la seguente porzione di codice e cosa stampa (non numericamente ma concettualmente).\\
\begin{minipage}[t]{0.5\linewidth}
\begin{lstlisting}[language=C]
#define elementi 5

int main() { 
int a[elementi],i,b; 
printf("Inserire %d numeri \n", elementi); 
for (i = 0; i < elementi; i++) 
  scanf("%d", &a[i]); 
for (i = 0; i < elementi/2; i++) { 
    b = a[i]; 
    a[i] = a[elementi-1-i]; 
    a[elementi-1-i] = b; 
    } 
for (i = 0; i < elementi; i++) 
  printf("%d %d\n", i, a[i]); 
}
\end{lstlisting}
\end{minipage}
\begin{minipage}[t]{0.5\linewidth}
  \makeemptybox{180pt}
\end{minipage}

%DOMANDA NUMERO 10

\question[2]
Assunto che venga inserito "9" cosa stamperà il codice ? e se venne inserito "f" ? di conseguenza a cosa serve la funzione "isdigit" ? \\
\begin{minipage}[t]{0.5\linewidth}
\begin{lstlisting}[language=C]
#include <stdio.h>
#include <ctype.h>
int main(void)
{
  char valore;
  printf("inserisci un carattere: ");
  valore = getchar();
  if (isdigit(valore)) {
    printf("si");
  }
  else {
    printf("no");
  }
  return 0;
}
\end{lstlisting}
\end{minipage}
\begin{minipage}[t]{0.5\linewidth}
  \makeemptybox{180pt}
\end{minipage}
\newpage
%DOMANDA NUMERO 11

\question[3]
Cosa stampa il seguente codice ?

\begin{lstlisting}[language=C]
int main(void)
{
  int a = 4;

  int b = a << 1;
  
  printf("a << 1 = %d\n", b);
}
\end{lstlisting}
\begin{oneparcheckboxes}
	\choice errore, in quanto << non è un operatore\\
	\choice a $<<$ 1 = a $<<$ 1 \\
	\choice a $<<$ 1 = 4 $<<$ 1 \\
	\choice a $<<$ 1 = 0,4\\
	\choice a $<<$ 1 = f\\
	\choice a $<<$ 1 = 8\\
	\choice a $<<$ 1 = 3\\
\end{oneparcheckboxes}

%DOMANDA NUMERO 12

\question[2]
Quale delle seguenti affermazioni è falsa?

\begin{oneparcheckboxes}
  \choice L’operatore "\&" applicato ad una variabile restituisce l’indirizzo di memoria in cui tale variabile e' allocata\\
  \choice L’operatore “*” applicato ad un puntatore restituisce il valore memorizzato in una locazione casuale\\
  \choice Se da una funzione f1 passo l’indirizzo di memoria di una variabile locale ad una funzione f2, allora f2 sarà
  in grado di modificare il valore di tale variabile\\
\end{oneparcheckboxes}

%DOMANDA NUMERO 13

\question[2]
Trovare gli errori nel seguente frammento di codice e riscriverli in forma corretta.

\begin{minipage}[t]{0.5\linewidth}
\begin{lstlisting}[language=C]
int main(void){
    float a, b ;
    float x ;
    printf("Equazione nella forma: ax+b=0\n");
    printf("Immetti coefficiente a:");
    scanf("%f", &a);
    printf("Immetti coefficiente b:");
    scanf("%f", &b);
    if( a != 0 ){
        x = - b / a;
        printf("La soluzione e' x=%d\n", x ;
    }
    else{
     if( b=0 ){
        printf("Equazione indeterminata\n");
                
     }else{
        printf("Equazione impossibil\n");
     }
   }
}
\end{lstlisting}
\end{minipage}
\begin{minipage}[t]{0.5\linewidth}
  \makeemptybox{225pt}
\end{minipage}
\newpage
%DOMANDA NUMERO 14

\question[2]
Qual è il numero massimo rappresentabile in complemento a due con 15 bit?.\\
\begin{minipage}[t]{1\linewidth}
	\makeemptybox{110pt}
\end{minipage}

%DOMANDA NUMERO 15

\question[2]
Scrivere un programma che determina se un numero intero inserito dall’utente è divisibile per 3.\\
\begin{minipage}[t]{1\linewidth}
	\makeemptybox{320pt}
\end{minipage}

\end{questions}
\end{document}










